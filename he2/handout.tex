%!TEX options=--shell-escape

\documentclass[fleqn,10pt]{olplainarticle}

\usepackage{hyperref}             % Clickable links
\usepackage{todonotes}            % Notes in margin
\usepackage{tikz}                 % Engraving figures
\usepackage[english]{babel}       % The theorem environment need something from here!
\usepackage{minted}               % A better verbatim.
\usepackage{enumitem}             % Resuming enumerations.
\usepackage{amsmath}              % Includes `\ensuremath` and friends.
\usepackage{amssymb}              % Symbols such as `\downarrow`.
\usepackage{stmaryrd}             % Symbols such as `\llbracket`.
\usepackage[nameinlink]{cleveref} % Better cross-referencing.
\usepackage{cite}                 % Citations.

% Additional macros.

\def\Name{\ensuremath{\text{\bf Name}}}
\def\OR{\ensuremath{\ |\ }}
\def\TO{\ensuremath{\rightarrow}}
\def\FROM{\ensuremath{\leftarrow}}
\def\LB{\ensuremath{\llbracket}}
\def\RB{\ensuremath{\rrbracket}}
\newcommand\semanticbracket[1]{\ensuremath{\LB{#1}\RB}}
\newcommand\LIT[1]{\ensuremath{\text{\tt #1}}}

\newcommand \BX [1]
  {\scriptsize\framebox{{\raisebox{0pt}[0.7\baselineskip][0.01\baselineskip]{\small #1}}}}

\newcommand\Axiom[2]
                 {\ensuremath{\text{\small #1}:\frac{\displaystyle}
                 {\displaystyle #2}}
                 }
\newcommand\InfOne[3]
                 {\ensuremath{\text{\small #2}:\frac{\displaystyle #1}
                 {\displaystyle #3}}
                 }
\newcommand\InfTwo[4]
                 {\ensuremath{\text{\small #3}:\frac{\displaystyle #1 \quad #2}
                 {\displaystyle #4}}
                 }
\newcommand\InfThree[5]
                 {\ensuremath{\text{\small #4}:
                     \frac{\displaystyle #1 \quad #2 \quad #3}
                          {\displaystyle #5}}
                 }
\newcommand\InfFour[6]
                 {\ensuremath{\text{\small #5}:
                     \frac{\displaystyle #1 \quad #2 \quad #3 \quad #4}
                          {\displaystyle #6}}
                 }
\newcommand\InfFive[7]
                 {\ensuremath{\text{\small #6}:
                     \frac{\displaystyle #1 \quad #2 \quad #3 \quad #4 \quad #5}
                          {\displaystyle #7}}
                 }
                % Judgement rules.
\renewcommand{\ref}[1]{\Cref{#1}} % Uses `cleverref` as ref.

% Use this macro instead of the language name.
% competing names are
% - CobraClass
% - Objective B.
% - HissScript.
% - BoaClass.
% - BoaByte (a byte code language for Boa)?
% - Boa++.
% - ViperScript.
% - NodeBoa.

\def\lang{Objective B}

\title{Implementing \lang}

\author[0]{Programming Language Implementation and Formalisation (Spring 2024)}

\begin{abstract}
The Cobra Cooperation has been receiving numerous error reports regarding
their imperative dynamically typed programming language, Boa. The primary
issue seems to stem from a lack of proper code structure and absence of a
type system, resulting in an increased number of errors. To address these
concerns, the Cobra Cooperation has initiated the development of a new
language, \lang, with the aim of resolving these issues and providing a more
efficient and error-free programming experience. In this assignment, you
will explore the design and features of \lang{ }and analyze the potential
impact it may have on the programming community. Additionally, you will have
the opportunity to propose potential improvements or additional features for
the language. Let's get started!
\end{abstract}

\begin{document}

\flushbottom
\maketitle
\thispagestyle{empty}

\section*{Syntax}

The goal of this assignment is to gain experience in implementing a
programming language from its formalisation. You will be implementing a
parser and interpreter for the \lang~programming language, for which you
will find the concrete syntax is defined by the BNF in
Figure~\ref{fig:syntax}.

\def\prog{\ensuremath{\mathbb{P}}}
\def\proc{\text{p}}
\def\type{\ensuremath{\tau}}
\def\exp{\text{e}}
\def\val{\text{v}}
\def\stmt{\text{s}}
\def\op{\ensuremath{\oplus}}
\newcommand\curlies[1]{\LIT{\{}#1\LIT{\}}}

\newcommand\IF[3]{\ensuremath{\LIT{if}~{#1}~\LIT{then}~{#2}~\LIT{else}~#3}}
\newcommand\ASS[3]{\ensuremath{#1~#2~\LIT{=}~#3}}
\newcommand\WHILE[2]{\ensuremath{\LIT{while}~#1~#2}}
\newcommand\RETURN[1]{\ensuremath{\LIT{return}~#1}}
\newcommand\PRINT[1]{\ensuremath{\LIT{print}(#1)}}
\newcommand\SEQ[2]{\ensuremath{#1~\LIT{;}~#2}}
\newcommand\APP[2]{\ensuremath{#1\LIT{(}#2\LIT{)}}}

\begin{figure}[ht!]
\begin{center}
\begin{align*}
  f, x &\in \Name   &\text{(Well-formed identifiers).}\\
  i, j &\in \mathbb{Z}                                      &\text{(Integers)}\\
\prog &:= \proc^*                                           &\text{(Programs)}\\
\proc &:= \type~f\LIT{(}\type_i~ x_i\LIT{)} \curlies{\stmt} &\text{(Procedures)}\\
\type &:= \LIT{integer}                                     &\text{(Primitive Types)}\\
      &\OR ~ \LIT{boolean}\\
      &\OR ~ \LIT{void}\\
\val  &:=  \LIT{true} \OR \LIT{false} \OR i \OR \LIT{null} &\text{(Simple values)}\\
\stmt &:=  \LIT{\{} \stmt \LIT{\}}                         &\text{(Statements)}\\
      &\OR ((\type)^?~x~\LIT{=})^?~\exp\\
      &\OR \IF{\exp}{\stmt_1}{\stmt_2}\\
      &\OR \WHILE{\exp}{\stmt}\\
      &\OR \RETURN{\exp}\\
      &\OR \SEQ{\stmt_1}{\stmt_2}\\
      &\OR \epsilon                                        &\text{(The empty statement)}\\
\exp  &:= \LIT{(} \exp \LIT{)}                             &\text{(Expressions)}\\
      %% &\OR \PRINT{\exp}\\
      &\OR \APP{f}{\exp_i}\\
      &\OR \exp_1 \op \exp_2\\
      &\OR \LIT{!} \exp\\
      &\OR x\\
      &\OR \val\\
\op   &:= \LIT{+}
       \OR \LIT{-}
       \OR \LIT{*}
       \OR \LIT{/}
       \OR \LIT{==}
       \OR \LIT{<}
       \OR \LIT{\&\&}
       \OR \LIT{||}                                        &\text{(Binary operators)}
\end{align*}
\end{center}
\vspace{-4mm}
\caption{The syntax of \lang. The star ($\odot^*$) denotes a sequence
  containing $0$ or more occurences. and the questionmark ($\odot^?$)
  denotes choice (zero or one), and the subscript ($\odot_i$) denotes a
  comma separated sequence containing zero or more indexable entities
  ($\odot_0, \odot_1, \dots, \odot_n$).}
\label{fig:syntax}
\end{figure}

A program \prog, consists of zero or more procedures, and a procedure is
defined by specifying its return type and name, followed by a list of
arguments and a statement. We will cover types and statements in more detail
later, but first, here are a couple of example procedures

\begin{minted}{text}
boolean isEven(integer n) {
  if   n == 0
  then return true
  else return isOdd(n - 1)
}

boolean isOdd(integer n) {
  if   n == 0
  then return false
  else return isEven (n - 1)
}

boolean isPrime(integer n){
  integer i = 2 ;
  while (i < n) {
    if (n / i) * i == n
    then return false
    else i = i + 1
  } ;
  return true
}
\end{minted}

As you may have noticed, the main part of a procedure is its statement.
To avoid syntactic ambiguity, statements can be grouped by curly braces.
For instance, the statement

\begin{minted}{text}
{ while e s1 } ; s2
\end{minted}

\noindent will run the statement \LIT{s1} until \LIT{e} holds, and then it will run
\texttt{s2} once. Whereas

\begin{minted}{text}
while e { s1 ; s2 }
\end{minted}

\noindent will run the statement \texttt{s1 ; s2} until \texttt{e} holds. Furthermore
a statement can be empty, and this is just to accomodate any leading or
trailing statement composiiton operators \SEQ{\stmt_1}{\stmt_2}.

\section*{semantics}

\def\src{\ensuremath{\pi}}
\def\state{\ensuremath{\sigma}}

\newcommand\runRWS[2]{\ensuremath{\langle #1 \rangle \downarrow \langle #2 \rangle}}
\newcommand\updatesState[3]{\ensuremath{\runRWS{\src, #1, #2}{ #3}}}

\newcommand\eval[2]{\runRWS{\src, \state, #1}{#2}}
\newcommand\lookup[3]{\runRWS{\src, #1, #2}{#3}}

A \lang~program must contain a procedure called \LIT{main}, and the
semantics of running a program, is to call \LIT{main} with its supplied
arguments. It is up to you to decide what those parameters should be, but
you must state your choice clearly in the report.

A procedure runs in the context of a source program \src, and a state
\state, containing the variable bindings currently in scope.

There are two evaluation judgements: An evaluation judgement for expressions
\eval{\exp}{\val} in which an expression evaluates to a value.
And, a judgement for statements, \runRWS{\src, \state_1, \stmt}{\val,
  \state_2}, in which a statement computes a value, and possibly changes the
state.
%
You can find the judgements for expressions and statements in
Figures~\ref{fig:evaluation-for-expressions} and
\ref{fig:evaluation-for-statements} respectively.

\begin{figure}[ht!]
  \begin{flushleft}
    \fbox{\runRWS{\src, \state, \exp}{\val}}
  \end{flushleft}
\begin{center}
  \Axiom
  {Variable}        % -------------------------------------
                    {\lookup{\state[x \mapsto v]}{x}{v}}
  \quad
  \Axiom
  {Value}           % -------------------------------------
                    {\eval{\val}{\val}}
  \\\vspace{3mm}
  \InfTwo           {\eval{\exp_i}{\val_i}}
                    {\runRWS{\src, [x_i \mapsto \val_i], \stmt}{\val, \state'}}
  {Call}            % -------------------------------------
                    {\runRWS{\src[\type f (\type_i, x_i) \curlies{\stmt}], \state
                                 , \APP{f}{\exp_i}}{\val}}
  \\\vspace{3mm}
  \InfOne           {\eval{\exp_1}{\LIT{true}}}
  {Not-True}        % -------------------------------------
                    {\eval{\LIT{!}\exp_1}{\LIT{false}}}
  \quad
  \InfOne           {\eval{\exp_1}{\LIT{false}}}
  {Not-False}       % -------------------------------------
                    {\eval{\LIT{!}\exp_1}{\LIT{true}}}
  \\\vspace{3mm}
  \InfTwo           {\eval{\exp_1}{\val_1}}
                    {\eval{\exp_2}{\val_2}}
  {Operation}       % -------------------------------------
                    {\eval{\exp_1 \op \exp_2}{\val}}
                    \hspace{-2mm}$(\val_1 \op \val_2 = \val)$
\end{center}
\vspace{-4mm}
\caption{}
\label{fig:evaluation-for-expressions}
\end{figure}

\begin{figure}[ht!]
  \begin{flushleft}
    \fbox{\runRWS{\src, \state_1, \exp}{\val, \state_2}}
  \end{flushleft}
\begin{center}
  \Axiom
  {Skip}            % -------------------------------------
                    {\runRWS{\src, \state, \epsilon}{\LIT{null}, \state}}
  \quad
  \InfOne           {\runRWS{\src, \state, \exp}{\val}}
  {Expression}      % -------------------------------------
                    {\runRWS{\src, \state, \exp}{\val, \state}}
  \quad
  \\\vspace{3mm}
  \InfOne           {\runRWS{\src, \state, \exp}{\val}}
  {AssignNew}       % -------------------------------------
                    {\runRWS{\src, \state, \type~x = \exp}{\val, \state[x \mapsto v]}}
  \quad
  \InfOne           {\runRWS{\src, \state, \exp}{\val}}
  {AssignOld}       % -------------------------------------
                    {\runRWS{\src, \state[x \mapsto w], x = \exp}{\val, \state[x \mapsto \val]}}
  \\\vspace{3mm}
  \InfTwo           {\runRWS{\src, \state_1, \exp}{\LIT{true}}}
                    {\runRWS{\src, \state_1, \stmt_1}{\val, \state_2}}
  {If-True}         % -------------------------------------
                    {\runRWS{\src, \state_1, \IF{\exp}{\stmt_1}{\stmt_2}}{\val, \state_2}}
  \\\vspace{3mm}
  \InfTwo           {\runRWS{\src, \state_1, \exp}{\LIT{false}}}
                    {\runRWS{\src, \state_1, \stmt_2}{\val, \state_2}}
  {If-False}        % -------------------------------------
                    {\runRWS{\src, \state_1, \IF{\exp}{\stmt_1}{\stmt_2}}{\val, \state_2}}
  \\\vspace{3mm}
  \InfThree         {\runRWS{\src, \state_1, \exp}{\LIT{true}}}
                    {\runRWS{\src, \state_1, \stmt}{\val_1, \state_2}}
                    {\runRWS{\src, \state_2, \WHILE{\exp}{\stmt}}{\val_2, \state_3}}
  {While-True}      % -------------------------------------
                    {\runRWS{\src, \state_1, \WHILE{\exp}{\stmt}}{\val_2, \state_3}}
  \\\vspace{3mm}
  \InfOne           {\runRWS{\src, \state, \exp}{\LIT{false}}}
  {While-False}     % -------------------------------------
                    {\runRWS{\src, \state, \WHILE{\exp}{\stmt}}{\LIT{null}, \state}}
  \quad
  \InfTwo           {\runRWS{\src, \state,   \stmt_1}{\val_1, \state_1}}
                    {\runRWS{\src, \state_1, \stmt_2}{\val_2, \state_2}}
  {Sequence}        % -------------------------------------
                    {\runRWS{\src, \state, \SEQ{\stmt_1}{\stmt_2}}{\val_2, \state_2}}
\end{center}
\vspace{-4mm}
\caption{}
\label{fig:evaluation-for-statements}
\end{figure}

% While will always return null
% Why do statements return a value if they can't be used as expressions?
    % Maybe for calls on functions that don't have an explicit return?
% Alternatively: Have all statements except return evaluate to the value null

\section*{Your task}

The board of Cobra Cooperation has requested that you write a prototype
interpreter for \lang. That is, your task is to initiate a \texttt{stack}
project, it could be called \texttt{objective-b}, and design and implement a
data structure and parser for the abstract syntax of \lang. Then you are
asked to write an interpreter for \lang. The choice is yours, and you can
take inspiration from \texttt{Boa} language. You should hand in a report
together with your code, describing the choices you made.

The board has a list of questions that you should answer in the report, and
a list of optional questions below.

\subsection*{Questions}
\begin{enumerate}
  \item Figure~\ref{fig:evaluation-for-statements} does not specify a
    semantics for the \texttt{return} statement. Suggest a judgement rule
    for \texttt{return} and include it in your implementation. Does your
    semantics for \texttt{return} affect the other judgements in
    Figure~\ref{fig:evaluation-for-statements}. If so, explain how your
    changes propagate.
  \item Cobra Cooperation has not yet designed a type system for \lang.
    Suggest a judgement form for typing expressions, and give two interesting example rules.
  \item How would you asses the current state of your prototype? How
    confident are you that the prototype is correct? What needs improvement
    in a production-ready interpreter?
\end{enumerate}

\subsection*{Challenges}
\begin{enumerate}
  \item \texttt{Boa} had a print expression. Extend your semantics with
    \texttt{print} and describe the choices you made. What happens to the
    evaluation rule for expressions?
  \item It is customary for procedural languages to provide a means of
    passing values between procedures. Extend the syntax of \lang~to
    accommodate pointers, implement pointers in your code.
  \item Extend \lang~with arrays.
  \item Extend \lang~with objects. What happens to your type system?
\end{enumerate}

\end{document}

